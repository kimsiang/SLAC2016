\chapter{Acronyms}

%%a add your acronyms here, e.g.
\begin{acronym}[RIKEN-RAL] % RIKEN-RAL is the longest acronym. Write the longest acronym symbol here on top for formatting reasons
%\acro{label}[symbol]{explanation}
\acro{qm}[QM]{quantum mechanics}
\acro{midas}[MIDAS]{Maximum Integrated Data Acquisition System}
\acro{truimf}[TRUIMF]{Canada's national laboratory for particle and nuclear physics and accelerator-based science}
\acro{qed}[QED]{quantum electrodynamics}
\acro{nmr}[NMR]{nuclear magnetic resonance}
\acro{qcd}[QCD]{Quantum Chromodynamics}
\acro{sm}[SM]{Standard Model}
\acro{hipa}[HIPA]{High Intensity Proton Accelerator}
\acro{fwhm}[FWHM]{Full Width at Half Maximum}
\acro{musr}[$\mu$SR]{Muon Spin Rotation}
\acro{msr}[MuSR]{Muonium Spin Rotation}
\acro{psi}[PSI]{Paul Scherrer Institute}
\acro{jparc}[J-PARC]{Japan Proton Accelerator Research Complex}
\acro{lem}[LEM]{Low Energy Muon}
\acro{lemusr}[LE-$\mu$SR]{Low Energy Muon Spin Rotation}
\acro{tf}[TF]{Transverse Field}
\acro{ethz}[ETHZ]{Eidgenössische Technische Hochschule Z{\"u}rich}
\acro{pst}[PST]{Positron Shielding Technique}
\acro{riken-ral}[RIKEN-RAL]{RIKEN-Rutherford Appleton Laboratory}
\acro{isis}[ISIS]{ISIS}
\acro{mcp}[MCP]{micro channel plate}
\acro{uhv}[UHV]{Ultra High Vacuum}
\acro{apd}[APD]{Avalanche Photo Diodes}
\acro{gui}[GUI]{Graphical User Interface}
\acro{hfs}[HFS]{hyperfine structure}
\end{acronym}

%% 
%\acro{label}[symbol]{explanation}
%
%Within the text you can use the command \ac{label} to call an acronym
%The first time you call an acronym it will appear as: "explanation (symbol)"
%from then on the command \ac{label} will produce: "symbol"
%
%the command \acf produces: "explanation (symbol)"
%the command \acs produces: "symbol"
%the command \acl produces: "explanation"
%
%These three commands may be useful if you want to customize the acronym behavior.
