\chapter{Standalone C++ Analysis framework}
\label{chap:four}

The package \verb+SLAC2016Ana+ contains an example \verb+C+++ framework to help you getting started. You can get it from the github.com by \verb+git clone https://github.com/kimsiang/SLAC2016+. You can build your additional analysis code on top of this example or write a new one based on it. The example is already running but it's not doing much yet. You can compile the analysis code by first sourcing your ROOT environment (e.g. \verb+source /usr/local/bin/thisroot.sh+) and then followed by executing the command \verb+make+.

This will read the necessary ingredients for compilation from the \verb+Makefile+ in the same directory. Don't have to worry much about this file at the moment unless you want to add in more classes to the analysis code.
The point is that it creates an executable named \verb+ana+. You can then execute the program by the command \verb+./ana input.script+ where \verb+input.script+ includes a path to the root file that you want to analyze (e.g. \verb+./test.root+). 

A description of the individual components of the example are given in the following list. Indicated are also the places where you should start adding your own code:

\begin{itemize}
\item  \verb+main.cxx+: This is the first starting point. It contains the \verb+main()+ function which is necessary for any \verb+C+++ program. The first step is to create instances of \verb+MyAna()+ class which is implemented in the files \verb+MyAna.h+ and \verb+MyAna.C+ (explained in the next items). The \verb+TChain+ represents the ROOT tree discussed in section 3. The files which should be read from disk are specified in the function \verb+Add(filename)+. The tree is then read and processed by the \verb+MyAna()+ class which takes the \verb+TChain+ as argument. The real work is then done in the \verb+Loop()+ function of the \verb+MyAna()+ class which is discussed in the next two items. 

\item \verb+MyAna.h+: Definition of the class \verb+MyAna+, which inherits from the \verb+TTree::MakeClass+. It declares variables and ROOT objects that will be used or stored in your analysis. Several basic functions that are common among event-based particle physics analysis like \verb+initialize()+, \verb+clear()+, \verb+execute()+ and\verb+finalize()+ are declared here.
%% to myself
%% need to change the names like Loop, execute, etc because it is a bit confusing
\item \verb+MyAna.C+: The main function which is called automatically which processing the ROOT trees are \verb+Loop()+. The \verb+Loop()+ function is called only once per run. In the \verb+Loop()+ function, \verb+initialize()+ is called at the beginning of the analysis run, \verb+clear()+ and \verb+execute()+ is called every event, and \verb+finalize()+ at the end of the analysis run.

\item \verb+t1.h+: Header file for the class \verb+t1+ created using \verb+TTree::MakeClass+.  

\item \verb+t1.C+: Source file for the class \verb+t1+ created using \verb+TTree::MakeClass+. The class \verb+Loop()+ is used by \verb+MyAna+ to loop through each \verb+TBranch+.

\item \verb+PlotAll.C+: A ROOT macro which can be used for automatic plotting of a set of histograms which are stored in a ROOT file. Please read the header of the file on how to use it. 
\end{itemize}
