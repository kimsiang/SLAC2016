\chapter{Get started}
\label{chap:getstarted}

\section{Installation}

This section is based on the official Offline Computing and Software Manual \url{http://gm2-docdb.fnal.gov:8080/cgi-bin/ShowDocument?docid=1825}.
Only important steps are reproduced here.

\subsection{Setting up the environment}

First of all you need to login to the Fermilab virtual machine (\verb+gm2gpvm02-04+).
Once you have logged in, you need to choose a release for the Muon g-2 softwares (libraries, executables)
every time you login or you can put it in your \verb+.profile+ on \verb+gm2gpvm+. Do the following:
%
\begin{Verbatim}[frame=single]
source /grid/fermiapp/gm2/setup
\end{Verbatim}
%
Alternatively if you want to run codes on your laptop, you can use cvmfs as mentioned in the documentation.
%
\begin{Verbatim}[frame=single]
source /cvmfs/gm2.opensciencegrid.org/prod/g-2/setup
\end{Verbatim}
%
If successful, you will get
%
\begin{Verbatim}[frame=single]
Using g-2 release repository at /cvmfs/gm2.opensciencegrid.org
g-2 software

--> To list gm2 releases, type
ups list -aK+ gm2

--> To use the latest release, do
setup gm2 v6_04_00 -q prof

For more information, see
https://cdcvs.fnal.gov/redmine/projects/g-2/wiki/ReleaseInformation.
\end{Verbatim}
%
Then follow the instruction and use the latest release of \verb+gm2+ (\verb+v6_04_00+ at the moment of writing this user guide).
Next to do is to go to a folder of your choice (usually \verb+~/work+) and then create a new development place.
%
\begin{Verbatim}[frame=single]
cd ~/work
mkdir SLACDev
\end{Verbatim}
%
Now go into the newly created folder \verb+SLACDev+ and initialize it as a new development area
%
\begin{Verbatim}[frame=single]
cd SLACDev/
mrb newDev
\end{Verbatim}
%
Follow the instruction to source the localproducts settings
%
\begin{Verbatim}[frame=single]
source localProducts_gm2_v6_04_00_prof/setup
\end{Verbatim}
%
Now we are ready to install new packages.

\subsection{Checkng out unpacking and reconstruction packages}
First we need to go into the \verb+srcs/+ folder to checkout the unpacking and reconstruction packages. 
Packages related to the data unpacking are \verb+gm2midas+, \verb+gm2midastoart+ and \verb+gm2unpackers+ whereas those related to the reconstruction are
\verb+gmcalo+. \verb+gm2dataproducts+ holds all the data structures for both.

Checkout (\verb+git+) \verb+gm2calo+ package using simplified command \verb+mrb g+.
%
\begin{Verbatim}[frame=single]
cd srcs/
mrb g gm2midas; mrb g gm2midastoart; mrb g gm2unpackers;
mrb g gm2calo; mrb g gm2dataproducts
\end{Verbatim}
%
For the package \verb+gm2midas+, you need to use the master branch,
%
\begin{Verbatim}[frame=single]
cd gm2midas
git checkout master
\end{Verbatim}
%
and build both \verb+midas+ and \verb+rome+:
%
\begin{Verbatim}[frame=single]
cd midas
export MIDASSYS=`pwd`
echo "export MIDASSYS=`pwd`" >> ~/.bash_profile
make; make install
cd ../rome
export ROMESYS=`pwd`
echo "export ROMESYS=`pwd`" >> ~/.bash_profile
make clean; make
\end{Verbatim}
%
Please refer to \url{https://cdcvs.fnal.gov/redmine/projects/gm2midas/wiki} for more detail.
Upon installing \verb+midas+ and \verb+rome+, go to \verb+gm2midastoart+ to build the midas file library provided by ROME,
%
\begin{Verbatim}[frame=single]
cd ../../gm2midastoart
git flow feature track SLAC2016
cd rome
make
\end{Verbatim}
%
For the packages \verb+gm2unpackers+, \verb+gm2calo+ and \verb+gm2dataproducts+
you need to use the SLAC2016 feature branch (you will be checking out develop branch if you do nothing)
%
\begin{Verbatim}[frame=single]
cd ../../gm2calo (gm2unpackers, gm2dataproducts)
git flow feature track SLAC2016
cd ..
\end{Verbatim}
%
Execute \verb+'mrb uc’+ to update the CMakeLists to include all the packages:
%
\begin{Verbatim}[frame=single]
mrb uc
\end{Verbatim}
%
Setup \verb+ninja+ (a small build system with focus on speed) with
%
\begin{Verbatim}[frame=single]
setup ninja v1_5_3a
\end{Verbatim}
%
Read more about it at \url{https://ninja-build.org/}.
Then start to build the packages with
%
\begin{Verbatim}[frame=single]
. mrb s
mrb b --generator ninja
\end{Verbatim}
%
If the build was successful (it takes about 5 minutes), you will see the following:
%
\begin{Verbatim}[frame=single]
------------------------------------
INFO: Stage build successful.
------------------------------------
\end{Verbatim}
%
Now set an alias for the ninja (because it is fast! except the first build though) build and put it in your \verb+.bash_profile+
%
\begin{Verbatim}[frame=single]
alias ninja='pushd $MRB_BUILDDIR; ninja; popd'
\end{Verbatim}
%
Now you are ready to unpack the data, reconstruct the data and analyze the data.

\subsection{Data unpacking and reconstruction}

The \verb+fcl+ files for the data unpacking are location in the \verb+gm2unpackers/fcl/+ folder,
whereas the \verb+fcl+ files for the data unpacking are location in the \verb+gm2calo/fcl/+ folder.
The script running  data unpacking and reconstruction are combined into a main \verb+fcl+ file named \verb+unpackFitConstants.fcl+.
Since we have several run periods based on their own calibration constants,
choose the nearest constant file which is below the run number of the file you want to analyze. For example, for run 1800, the nearest
constant file is \verb+constants1751.fcl+. However, \verb+unpackFitConstants1751.fcl+ is available so you do not have to worry about creating one yourself.
To analyze any midas file you like, simply
%
\begin{Verbatim}[frame=single]
gm2 -c unpackFitConstants1751.fcl -s run01800.mid -o gm2slac_run01800.art
-T gm2slac_run01800.root
\end{Verbatim}
%
where \verb+-o+ is followed by the location of the \textit{art} file and \verb+-T+ is followed by the location of the root file you want store.
Even better, an automation script, \verb+auto_unpack_fit.sh+, where the calibration period is taken care of, is also available. Simply run this script with the
following command:
%
\begin{Verbatim}[frame=single]
./auto_unpack_fit.sh startRunNum endRunNum
\end{Verbatim}
%
provided you have set the paths in the bash script like \verb+midas_dir+, \verb+art_dir+ and \verb+root_dir+ correctly.
