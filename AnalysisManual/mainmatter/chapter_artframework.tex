\chapter{Data analysis framework}
\label{chap:analysisframework}

\section{Offline framework for the SLAC test beam}

\ac{midas} DAQ system is developed at \ac{psi} and \ac{truimf}.

\begin{figure}[htbp]
\centering
%\fbox{\includegraphics[trim=0cm 5.5cm 0cm 5.5cm ,width=0.9\textwidth]{pics/EMShower}} guide line for trimming
\includegraphics[width=0.7\textwidth]{pics/offline_exp_framework}
\caption{An overview of the Muon g-2 offline framework.}
\label{pic:exp_framework}
\end{figure}

As shown in \fref{pic:exp_framework}, data analysis for this test beam has several components. First we need to convert the raw data stored in a MIDAS file (\verb+.mid+ or \verb+.mid.gz+) to \textit{art} data products stored in an \textit{art} file.
This is handled using \textit{art} framework's modules and is doing nothing more than storing \verb+16-bit+ or \verb+32-bit+ word into \verb+vectors+. Next we unpack these \verb+vectors+
and give them contexts based on the header information stored within the \verb+vectors+. At this step, all the information are stored as data products you are probably familiar with: \verb+RiderArtRecord+,
\verb+IslandArtRecord+, etc. Then reconstruction algorithms are ran through these data products and at the end of the chain each physics objects are reconstructed as clusters.

\begin{figure}[htbp]
\centering
\includegraphics[width=0.7\textwidth]{pics/offline_slac_framework}
\caption{Offline framework for the SLAC experimental data.}
\end{figure}

\subsection{Data Acquisition}
Explain the DAQ flow here roughly (from machine trigger to FC7, from FC7 fanout to AMC13 of all the crates, then from AMC13 to AMCs in a crate).


\subsection{Data Unpacking}
The task of unpacking the raw midas data is divided into several \textit{art} modules. First, the Midas banks are converted into TBranches as vectors in the source input module under the repository gm2midastoart.
Then the header information stored in the \verb+CB+ banks is unpacked by the module \verb+HeaderUnpacker+, 
the raw waveforms stored in the \verb+CR+ banks in the calorimeter (fc7, laser) crate is unpacked by the RawUnpacker (LaserRawUnpacker, FC7Unpacker), the chopped islands stored in the \verb+CT+ banks
is unpacked by the IslandUnpacker (LaserUnpacker) and so on. Fhicl file configuration will be explained in the next section.

\subsection{Reconstruction}
The data reconstruction chain is consisted of 5 modules in series: pulse fitter, energy calibrator, gain corrector, time corrector and hit cluster. Fhicl file configuration will be explained in the next section.