\chapter{T-536 program}
\label{app:program}

This chapter contains the full program we have planned and executed during the SLAC run.

\begin{landscape}

\begin{longtable}{|p{8cm}|p{12cm}|} \hline
Study &  Comment / Instruction / Run Numbers \\ \hline
Vertical Sweep & 1649 - 1653 \\ \hline
Calibration Sequnce & Runs: 1665 - 1672 \\ \hline
Template xtal 24 & Runs: 1673 - ... one hour \\ \hline
Test PIX image of beam & (failed so far) \\ \hline
crystal centers in the \newline cluster of 4x4 crystals: & 2000 events per run \\ \hline
34, 33, 32, 31 \newline 25, 24, 23, 22 \newline 16,15,14,13 & use relPos buttons to move \newline red means done \\ \hline
high statistics seg 24 & 1 hour \\ \hline
Fine position scan of segment 24 & for y in 0, and 10 \newline for x from -32 to 32 in steps of 4  \newline (64 mm scanned out of 64 for y=0) \newline Current position x=277, y=95, Xtal 23 \newline 3000 events per run \newline total run time: 6 hours \\ \hline
Cross of 4 segments & Intersection of segment 24, 25, 33, 34  \newline Intersection of segment 25, 26, 34, 35 \\ \hline
Edge scan outward of segment 26 & For y=0, 6 points from center to calo edge \newline For y=10, 6 points from center to calo edge \newline Remain to be done: x=\{332.6,330.1,327.6\}, y=95 \\ \hline
Crack scan & Crack 24-25 ; 25-26 ; fine for 24-25-33-34 \\ \hline
Energy scan at 3 points for each \newline 3.5 GeV \newline 4.0 GeV \newline 4.5 GeV  \newline 5.0 GeV \newline 2.5 GeV & 4 X,Y Locations at each energy  \newline (277,105) (291.8,105) (304.4,105) (291.8,92.4) \newline (279.2,105) (291.8,105) (304.4,105) (291.8,92.4) \newline (279.2105) (291.8,105) (304.4,105) (291.8,92.4) \newline (279.2,105) (291.8,105) (304.4,105) (291.8,92.4) \newline (279.2,105) (291.8,105) (304.4,105) (291.8,92.4) \newline (279.2,105) \\ \hline
Edge scan outward of segment 26 & For y=95, x=347.45 (outside); 344.95 (outside); \newline  342.45 ; 339.95 ; 337.45; 334.95; 332.45; 329.95 \newline Repeat for y=105, x=347.45 (outside); 344.95 (outside); \newline  342.45 ; 339.95 ; 337.45; 334.95; 332.45; 329.95 \\ \hline
Angle Scan at 5 degrees & 3000 events per step Y = 105; X =  \newline {[}284, 288, 292, 296, 300, 304, 308, 312, 316, 320, \newline 324, 328, 332, 336, 340, 344, 348, 352, 356, 360{]} \newline 3000 events per step Y = 95; X = \newline  {[}284, 288, 292, 296, 300, 304, 308, 312, 316, 320, \newline 324, 328, 332, 336, 340, 344, 348, 352, 356, 360{]} \\ \hline
Angle Scan at 10 degrees & 3000 events per step Y = 105; X = \newline {[}286, 290, 294, 298, 302, 306, 310, 314, 318, 322, \newline 326, 330, 334, 338, 342, 346, 350, 354, 358, 362{]} \newline 3000 events per step Y = 95; X = \newline {[}286, 290, 294, 298, 302, 306, 310, 314, 318, 322, \newline 326, 330, 334, 338, 342, 346, 350, 354, 358, 362{]} \newline then a long run at Y = 95, X = 306 till 9AM \\ \hline
Angle Scan at 15 degrees & 3000 events per step Y = 105; X = \newline {[}294, 298, 302, 306, 310, 314, 318, 322, 326, 330, \newline 334, 338, 342, 346, 350, 354, 358, 362, 366, 370{]} \newline 3000 events per step Y = 95; X = \newline {[}294, 298, 302, 306, 310, 314, 318, 322, 326, 330, \newline 334, 338, 342, 346, 350, 354, 358, 362, 366, 370{]} \newline then a long run at Y = 95, X = 310 till 9PM \newline with filter wheel calibration in parallel \\ \hline
Moving Flight Simulator Tests (1) & FSrate=32; FWstate=6; 1/2 mode; \newline T0 = -110950 (start of exp sequence) \newline scan the first 20 usec, 3000 events per run \newline trigger delay -120950, -118950, -116950,  \newline -114950, -112950, -110950, -120950, -122950, \newline  -124950, -126950, -128950, -130950, \newline  scan the first 100 usec, 3000 events per run \newline trigger delay -130950, -140950, -150950, \newline  -160950, -170950, -190950, -210950 \newline 60 min runs \newline Delay = 15 us, 30 us, 60 us, 120 us, 240 us, \newline  480 us; Check statistics along way for run \newline length precision in ratio of on/off \\ \hline
Moving Flight Simulator Tests (2) & FSrate=32; FWstate=5; 1/2 mode; \newline T0 = -110950 (start of exp sequence) \newline scan the first 20 usec, 3000 events per run \newline trigger delay -120950, -118950, -116950, \newline  -114950, -112950, -110950, -120950, -122950, \newline  -124950, -126950, -128950, -130950, \newline  scan the first 100 usec, 3000 events per run \newline trigger delay -130950, -140950, -150950, \newline  -160950, -170950, -190950, -210950 \\ \hline
Moving Flight Simulator Tests (3) & FSrate=32; FWstate=1; 1/2 mode; \newline T0 = -110950 (start of exp sequence) \newline scan the first 20 usec, 6000 events per run \newline trigger delay -120950, -116950, -112950 (stopped), \newline  -124950, -130950, -140950, -160950 \\ \hline
Fine vertical scan seg 24 & 3000 events per run at X = 284, and Y = \newline {[}85, 89, 93, 97, 101, 105, 109, 113, 117, 121, 125{]} \\ \hline
Horizontal scan seg 24 & 3000 events per run at Y = 105 and X = 280, 284, 288 \\ \hline
Front face scan & widen island for template gen (100 pre, 200 post) \newline 4500 events per segment \newline 26, 25, 24, 23, 22, 21, 20, 19, 18, \newline 9, 10, 11, 12, 13, 14, 15, 16, 17, \newline 35, 34, 33, 32, 31, 30, 29, 28, 27, \newline 36, 37, 38, 39, 40, 41, 42, 43, 44, \newline 53, 52, 51, 50, 49, 48, 47, 46, 45, \newline 0, 1, 2, 3, 4, 5, 6, 7, 8 \\ \hline
undo wide islands for templates back to 8 pre-trigger, and 24 post-trigger & \\ \hline
Flight sim test, FW\#1, 96 shots per fill & FSrate=96; FWstate=1; 1/2 mode; \newline 9000 events per run \newline trigger delay -120950, -118950, -116950,  \newline -114950, -112950, -122950, -124950,  \newline -126950, -128950, -130950, \\ \hline
redo ping seg 8, 3 & switch laser to 100kHz fixed \newline widen islands for template generation \newline collect 4500 events \newline undo wide islands \\ \hline
low QE run & be creative \\ \hline
long run & 25 deg, two diff positions, same beam, \newline 6 hours each, filter wheel calibration before, after and during \\ \hline
position scan for 25 deg & 9000 events per point Y = 105, X =  \newline 284.0, 288.0, 292.0, 296.0, 300.0, \newline 304.0, 308.0, 312.0, 316.0, 320.0, \newline 324.0, 328.0, 332.0, 336.0, 340.0, \newline 344.0, 348.0, 352.0, 356.0, 360.0, \newline 364.0, 368.0, 372.0, 376.0, 380.0 \newline 9000 events per point Y = 95, X =  \newline 284.0, 288.0, 292.0, 296.0, 300.0, \newline 304.0, 308.0, 312.0, 316.0, 320.0, \newline 324.0, 328.0, 332.0, 336.0, 340.0, \newline 344.0, 348.0, 352.0, 356.0, 360.0, \newline 364.0, 368.0, 372.0, 376.0, 380.0 \\ \hline
horizontal scan of fiber harp & 3000 events per run \newline Y = 114, 110, 106, 102, 98 \newline X = 240, 244, 248, 252, 256, \newline 260, 264, 268, 272, 276, \newline 280, 284, 288, 292, 296 \newline 300, 304, 308, 312, 316, \newline 320, 324, 328, 332, 336, \newline 340, 344, 348, 352, 356, 360 \\ \hline
horizontal scan of fiber harp in calibration position, vertically centered in beam & 3000 events per run \newline Y = 105 X = 240, 244, 248, 252, 256, \newline 260, 264, 268, 272, 276, \newline 280, 284, 288, 292, 296, \newline 300, 304, 308, 312, 316, \newline 320, 324, 328, 332, 336, \newline 340, 344, 348, 352, 356, 360 \\ \hline
scan of intensity and bias voltage & at Y = 105, X = 260 mm \newline bias voltages = 65.5, 66.0, 66.5, 67.0, 67.5, 68.0, 68.5  \newline VSL10 = 20, 15, 10, 5 mm \newline C24\_H = 1, 2, 4 mm \newline 6000 events if SL10*C24\_H \textless= 10 mm\textasciicircum 2; otherwise 3000 events \\ \hline
horizontal scan of fiber harp in calibration position with narrow beam & 3000 events per run \newline Y = 105, X = \newline 240, 244, 248, 252, 256, \newline 260, 264, 268, 272, 276, \newline 280, 284, 288, 292, 296,  \newline 300, 304, 308, 312, 316, \newline 320, 324, 328, 332, 336, \newline 340, 344, 348, 352, 356, 360, X = \newline 242, 246, 250, 254, 258, \newline 262, 266, 270, 274, 278,  \newline 282, 286, 290, 294, 298,  \newline 302, 306, 310, 314, 318, \newline 322, 326, 330, 334, 338,  \newline 342, 346, 350, 354, 358 \\ \hline
long run & 1 hour = 36,000 events total  \newline 6 runs of 6000 events each (10 min)  \newline Y = 105, X = 260  \newline SL10 = 20, C24 = 1.5 \\ \hline
bias voltage scan & 3000 events,  \newline Y = 105, X = 260,  \newline SL10 = 20, C24 = 1.5,  \newline bias voltages = 65.5, 66.0, 66.5,  \newline 67.0, 67.5, 68.0, 68.5 V \\ \hline
horizontal scan of fiber harp in calibration position, new beam width & 3000 events per run \newline SL10 = 20 mm, C24 = 1.5 mm,  \newline Y = 105, X = \newline  240, 244, 248, 252, 256, \newline 260, 264, 268, 272, 276, \newline  280, 284, 288, 292, 296,  \newline 300, 304, 308, 312, 316,  \newline 320, 324, 328, 332, 336, \newline  340, 344, 348, 352, 356, 360 \\ \hline
scan of intensity and bias voltage at ideal position along fibers & at Y = 105, X = 276 mm \newline bias voltages = 65.5, 66.5, 67.5, 68.5 V, \newline  C24\_H = 1.5, 2, 4 mm \newline SL10 = 20, 10, 5 mm, \newline  3000 events \\ \hline
scan of intensity and bias voltage at ideal position along fibers & at Y = 105, X = 276 mm \newline bias voltages = 66, 67, 68 V  \newline (at C24 = 2, SL10 = 20, \newline  do more points: 65.5, 66,  \newline 66.5, 67, 67.5, 68, 68.5 V) \newline C24\_H = 1.5, 2, 4 mm \newline SL10 = 20, 10, 5 mm \newline 3000 events \\ \hline
long run at ideal position along fibers & 1 hour = 36,000 events total  \newline 6 runs of 6000 events each (10 min) \newline Y = 105, X = 276 \newline SL10 = 20, C24 = 2 \\ \hline
long run with white paint on fiber ends & 1 hour = 36,000 events total  \newline 6 runs of 6000 events each (10 min) \newline Y = 105, X = 276 \newline SL10 = 20, C24 = 2 \\ \hline
horizontal scan of fiber harp in calibration position with white paint & 3000 events per run  \newline SL10 = 20 mm, C24 = 2 mm,  \newline Y = 105, X = \newline 264, 268, 272, 276, 280, \newline  284, 288, 292, 296, 300, \newline  304, 308, 312, 316, 320,  \newline 324, 328, 332, 336, 340, \newline  344, 348, 352, 356 \\ \hline
vertical scan of T0 counter & 3000 events per run  \newline SL10 = 20 mm, C24 = 2 mm \newline X = 288 \newline Y = 65, 75, 85, 95, (105), \newline  115, 125, 135, 145 \\ \hline
laser before beam at different time separations & 3000 events, \newline  Y = 105, X = 276,  \newline SL10 = 20, C24 = 2,  \newline deltaT = 10, 20, 30, 40, 50, 75, 100, 150, 300 ns \\ \hline
\end{longtable}



\end{landscape}
