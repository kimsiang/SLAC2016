\documentclass[12pt,letterpaper]{article}
\usepackage[utf8]{inputenc}
\usepackage{amsmath}
\usepackage{amsfonts}
\usepackage{amssymb}
\usepackage{graphicx}
\usepackage[left=2cm,right=2cm,top=2cm,bottom=2cm]{geometry}
\title{Timing alignment for a calorimeter/uTCA crate}
\author{Kim Siang Khaw}

\begin{document}
\maketitle
\abstract{This article summarizes a strategy that can be used to time-align all the 54 rider channels within a calorimeter.
Two types of signal can be used to time-align the rider channels within a calorimeter.
First is the laser sync pulse and second is the electron/positron beam. This article is intended to show how they can be utilized to do so.
}

\section{Basics}

In general, the time extracted from the template fit of a sync pulse for crystal $i$ and fill $f$ can be written as
%
\begin{equation}
t_{sync}(i,f) = t_{diffuser}(f) + \tau_{fiber}(i) + \tau_{rider}(i,f) \label{eq:tsync}
\end{equation}
%
where $t_{laser}(f)$ is the time at which the laser system is triggered relative to the rider crate trigger, $\tau_{fiber}(i)$ the laser propagation time in a fiber $i$ and $\tau_{rider}$ the internal delay of a rider channel and includes the 2 clock tick shift. Likewise, for the electron/positron beam,
%
\begin{equation}
t_{beam}(i,f) = t_{calohit}(f)  + \tau_{shower}(i,f) + \tau_{rider}(i,f)\label{eq:tbeam}
\end{equation}
%
where $t_{calohit}(f)$ is the hit time of the beam on the front face of the calorimeter relative to the rider crate trigger, $\tau_{shower}(i,f)$ is the EM shower plus the mean Cherenkov light propagation time to the SiPM $i$.

\subsection*{Alignment using sync pulse}
During the SLAC test run, we have done some analyses using the sync pulse for timing alignment.
This means the synced time for the beam event is
%
\begin{align}
t_{beam,synced}(i,f) &=  t_{beam}(i,f) - t_{sync}(i,f) \\
                               &= t_{calohit}(f) - t_{laser}(f) + \tau_{shower}(i,f) - \tau_{fiber}(i)\label{eq:tbeamsync}
\end{align}

The timing difference $\delta t$ between 2 crystals ($i$ and $j$) in fill $f$ is thus given by
%
\begin{align}
\delta t_{beam,synced}(i,j,f) &=   t_{beam,synced}(i,f) - t_{beam,synced}(j,f) \\
                               &= \delta\tau_{shower}(i,j,f) - \delta\tau_{fiber}(i,j) \label{eq:dtbeamsync}
\end{align}
%
since $t_{calohit}(f) - t_{laser}(f)$ does not depend on the crystal. 
%
It is obvious from Eq.~\ref{eq:dtbeamsync} that due to inhomogeneous fiber lengths, sync pulses alone are not enough for the timing alignment.
Hence we need to find a way to extract the propagation time in the fiber, $t_{fiber}$, for all 54 fibers. Bad news is, it is not trivia to measure each fiber's propagation time up to 100~ps precision. Good news is, we just need to measure the relative difference within a calorimeter. For historical reason,
we have selected the crystal 24's rider channel as the reference point.

\subsection*{Alignment using sync pulse and beam pulse}

Aligning all the timing information w.r.t to the crystal 24, we have

\begin{align}
\delta t_{sync}(i,24,f)   &= t_{sync}(i,f) - t_{sync}(24,f)  \\
                         &= \left[\tau_{fiber}(i) -  \tau_{fiber}(24) \right]+ \left[\tau_{rider}(i,f) -  \tau_{rider}(24,f) \right] \\
                         &= \delta \tau_{fiber}(i,24) + \delta \tau_{rider}(i,24,f) \label{eq:tsync24}
\end{align}
%
and
%
\begin{align}
\delta t_{beam}(i,24,f)  &= t_{beam}(i,f) - t_{beam}(24,f)  \\
                         &= \left[ \tau_{shower}(i,f) -  \tau_{shower}(24,f)  \right] + \left[ \tau_{rider}(i,f) -  \tau_{rider}(24,f) \right] \\
                         &= \delta \tau_{shower}(i,24,f) + \delta \tau_{rider}(i,24,f)~.\label{eq:tbeam24}
\end{align}

If we choose all the events such that the shower propagation time in the crystal is the same w.r.t. to the crystal 24, i.e. $\delta \tau_{shower}(i,24,f)=0$ (for example, beam hitting the center of the crystal or hitting at the common border of the crystals),
then we have
%
\begin{equation}
\delta t_{beam}(i,24,f)   = \delta \tau_{rider}(i,24,f)~.
\end{equation}
%
Since the offset in the rider $\tau_{rider}$ is the same for all the fills within the same run, 
%
\begin{equation}
\delta t_{beam}(i,24)   = \delta \tau_{rider}(i,24)~.
\end{equation}
%
Of course, since the shower is well contained within the $3\times3$ crystals, it is not possible to map out all the $\delta \tau_{rider}$ by just analyzing
the data w.r.t. to the crystal 24. Hence to have a complete map, we need to look at the $\delta \tau_{rider}$ w.r.t. to the crystals that have the  $\delta \tau_{rider}$ extracted compared to crystal 24. The statistical uncertain will increase but there should be an optimum way of doing it out there.
%
Once we have the map of the $\delta \tau_{rider}$, we can solve Eq.~\ref{eq:tsync24} and get $\delta \tau_{fiber}$.
If we turn things around, since we have all the $\delta \tau$ relative to crystal 24, we can also say that we have all the $\delta \tau$
relative to each other.

Going back to Eq.~\ref{eq:dtbeamsync}, since we have the $\delta \tau_{fiber}$, we now have a better knowledge 
regarding $\tau_{shower}$. We can now discriminate more pile up events with this additional information.
We might also be able to disentangle the position and the angle information.

\section{Datasets}

\section{Analysis}


\subsection{How often does the $\delta \tau_{rider}$ changes?}

\end{document}
